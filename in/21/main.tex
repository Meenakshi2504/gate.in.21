\let\negmedspace\undefined
\let\negthickspace\undefined
\documentclass[journal,12pt,twocolumn]{IEEEtran}
\usepackage{cite}
\usepackage{amsmath,amssymb,amsfonts,amsthm}
\usepackage{algorithmic}
\usepackage{graphicx}
\usepackage{textcomp}
\usepackage{xcolor}
\usepackage{txfonts}
\usepackage{listings}
\usepackage{enumitem}
\usepackage{mathtools}
\usepackage{gensymb}
\usepackage{comment}
\usepackage[breaklinks=true]{hyperref}
\usepackage{tkz-euclide} 
\usepackage{listings}
\usepackage{gvv}                                        
\def\inputGnumericTable{}                                 
\usepackage[latin1]{inputenc}                                
\usepackage{color}                                            
\usepackage{array}                                            
\usepackage{longtable}                                       
\usepackage{calc}                                             
\usepackage{multirow}                                         
\usepackage{hhline}                                           
\usepackage{ifthen}                                           
\usepackage{lscape}

\newtheorem{theorem}{Theorem}[section]
\newtheorem{problem}{Problem}
\newtheorem{proposition}{Proposition}[section]
\newtheorem{lemma}{Lemma}[section]
\newtheorem{corollary}[theorem]{Corollary}
\newtheorem{example}{Example}[section]
\newtheorem{definition}[problem]{Definition}
\newcommand{\BEQA}{\begin{eqnarray}}
\newcommand{\EEQA}{\end{eqnarray}}
\newcommand{\system}[1]{\stackrel{#1}{\rightarrow}}
\newcommand{\define}{\stackrel{\triangle}{=}}
\theoremstyle{remark}
\newtheorem{rem}{Remark}
\begin{document}

\bibliographystyle{IEEEtran}
\vspace{3cm}

\title{12.10.4}
\author{EE22BTECH11008 - Annapureddy Siva Meenakshi$^{*}$% <-this % stops a space
}
\maketitle
\bigskip

\renewcommand{\thefigure}{\theenumi}
\renewcommand{\thetable}{\theenumi}
Q:A system has transfer function
 \[\frac{Y(s)}{X(s)}=\frac {s-\pi}{s+\pi}\]
 let $u(t)$ be the unit step function.The input $x(t)$ that results in a steady-state output $y(t)=sin(\pi t)$ is \underline{\quad}.
\solution
\begin{table}[!ht]
    \centering
        \begin{tabular}{|c|c|c|} 
    \hline
    \textbf{Variable} & \textbf{Description} & \textbf{Value} \\
    \hline
    $x(t)$ & input function & none \\
    \hline
    $y(t)$ & output function & $\sin(\pi t)$ \\
    \hline
    $H(s)$ & Transfer-function & $\frac{s-\pi}{s+\pi}$ \\
    \hline
\end{tabular}

    \caption{input parameters}
    \label{tab:in_21_t1}
\end{table}
 \begin{equation}
     H(s)=\frac{s-\pi}{s+\pi}
 \end{equation} 
 Converting transfer function to frequency response, we get
 \begin{equation}
     H(j\omega)=\frac{j\omega-\pi}{j\omega+\pi}
 \end{equation}
 Here , $\omega=\pi$
 \begin{align}
    H(j\pi)&=\frac{j-1}{j+1}=j\\
     |H(j\pi)|&=1\\
    \angle H(j\pi)&=90^\circ\\
    y(t)&=\sin(\pi t)=\cos\left(\pi t+\frac{3\pi}{2}\right)\label{eq:in.21.6} 
\end{align}
\begin{align}
  |X|\cos(\omega t+\phi)&\system{H(j\omega)}|X||H(j\omega)|\cos(\omega t +\phi +\angle H(j\omega))\label{eq:in.21.7}\\
  \cos(\pi t+\pi)&\system{H(j\omega)}|H(j\omega)|\sin(\pi t +\pi +\angle H(j\omega))
\end{align}
Therefore by ~\eqref{eq:in.21.6} and ~\eqref{eq:in.21.7} , we get
\begin{align}
    x(t)&=cos(\pi t+\pi)\\
    &=\sin\left(\pi t -\frac{\pi}{2}\right)
\end{align}
 \end{document}
